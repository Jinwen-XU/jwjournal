\documentclass[11pt, paperstyle=light yellow, color entry]{jwjournal}
% "paperstyle = ..." ajuste la couleur du papier, les options incluent : lightyellow、yellow、parchment、green、lightgray、gray、nord、dark
% "color entry" ajoute plus de couleur au titre de chaque entrée
% "scroll" active le mode de défilement et peut générer un pdf d'une seule page similaire à une longue capture d'écran

\UseLanguage { French }

\begin{document}


2023-01-01 Ensoleillé --- Appartement

  Les dates et annotations apparaissant dans le texte ne seront pas reconnues : 2022-02-02, [Annotation], alors n'hésitez pas à les écrire.

  [Sports] quelque chose sur les événements sportifs

  [Apprendre] quelque chose sur l'apprentissage



2023-01-02
Inconnu (la météo peut aussi être écrite sur la deuxième ligne, selon votre préférence)

  La couleur des étiquettes changera en fonction de la date.

  [Note] Je n'ai pas étudié aujourd'hui, mais j'ai écrit beaucoup beaucoup beaucoup beaucoup beaucoup beaucoup beaucoup beaucoup beaucoup de code.



2023-01-03

  [Note] Si seule la date est écrite, il n'y aura pas de deuxième séparateur après celle-ci.


+++
2023-01-04  Nuageux

  Trois signes plus \texttt{+++} peuvent agrandir la page courante d'environ une ligne. Parfois, une seule phrase ou quelques mots tombent sur la page suivante --- dans de tels cas, vous pouvez l'utiliser pour améliorer l'effet d'affichage.



2023-01-05  Nuageux

  ...



2023-01-06  Nuageux

  ...



2023-01-07  Nuageux

  ...



2023-01-08  Nuageux

  Une semaine plus tard, la couleur revient à la précédente...



2023-01-09  Nuageux

  ...


\end{document}
