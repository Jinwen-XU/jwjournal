\documentclass[11pt, paperstyle=light yellow, color entry, day-month-year,
  title in boldface, title in sffamily, use style = classical]{jwjournal}
% "paperstyle = ..." ajuste la couleur du papier, les options incluent : lightyellow、yellow、parchment、green、lightgray、gray、nord、dark
% "color entry" ajoute plus de couleur au titre de chaque entrée
% "scroll" active le mode de défilement et peut générer un pdf d'une seule page similaire à une longue capture d'écran

\UseLanguage { French }

\begin{document}


01/01/2025 Ensoleillé | Lieu

  Les dates et annotations apparaissant dans le texte ne seront pas détectées : 25/12/2024, [Note], alors n'hésitez pas à les écrire !

  [Texte]
    Vous pouvez écrire votre texte en *italique*, **gras** ou ***les deux***.
    \\
    Les astérisques en mode math ne seront pas détectées : $a^* b^*$.

  [Remarque] Dans certains cas particuliers, il serait nécessaire de séparer les astérisques par~\texttt{\{\}} : **comme**{}***cet***{}*exemple*.

  >>> Vous pouvez mettre votre texte dans un bloc coloré comme celui-ci.

  +++
  Les images peuvent être incluses via :

  || <.4> {example-image} % ou || {example-image} <.4>
  >> (Remarque sur l'image)
  >> (Plus de remarque...)

  Vous pouvez également afficher l'image à gauche ou à droite :
  (( <.4> {example-image-a} % ou (( {example-image-a} <.4>
  -> <.3> % ajouter un certain espace vertical, ici c'est 0.3\baselineskip
  )) <.4> {example-image-b} % ou )) {example-image-b} <.4>


## {\textsc{Une Nouvelle Section}}

02/01/2025
Inconnu (la météo peut aussi être écrite sur la deuxième ligne, selon votre préférence)

  La couleur des étiquettes changera en fonction de la date, et le texte long sera automatiquement indenté :

  [Note] Je n'ai pas étudié aujourd'hui, mais j'ai écrit beaucoup beaucoup beaucoup beaucoup beaucoup beaucoup beaucoup de code: `code en ligne`.

  [Attention] Les caractères spéciaux dans le code en ligne doivent être échappés. Par exemple, `\textbackslash` doit être écrit comme `\textbackslash textbackslash`, `\{` doit être écrit comme `\textbackslash\{`, `\%` doit être écrit comme `\textbackslash\%`, etc.

  -----

  Trois ou plus \textquote{\texttt{-}} signes dans un paragraphe isolé vous donnent une ligne de séparation ---

  ---

  --- mais la ligne actuelle ne sera pas prise en compte car il y a du texte supplémentaire derrière \texttt{-}\texttt{-}\texttt{-}.

  -----

  Si l'étiquette est suivie de plusieurs paragraphes, vous pouvez utiliser \texttt{\textbackslash\textbackslash} et \texttt{\slash\slash} judicieusement pour les concaténer, de manière à ce que tous les textes soient indentés convenablement :

  [Étiquette] Un peu de texte.
    //
    Plus de texte (avec \texttt{\slash\slash}, ce qui permet d'avoir un espace vertical par rapport au texte ci-dessus).
    //
    Plus de texte (le même, avec un certain espace vertical par rapport au texte précédent).
    //
    (Quelque remarque)
    \\
    (Une autre remarque, due à l'utilisation de \texttt{\textbackslash\textbackslash}, est proche de la remarque ci-dessus)
    \\
    (Une autre remarque --- de même, est proche de la remarque ci-dessus)



03/01/2025

  [Note] Si seule la date est écrite, il n'y aura pas de deuxième séparateur après celle-ci.


+++
04/01/2025  Nuageux

  Trois signes plus \texttt{+++} peuvent agrandir la page courante d'environ une ligne. Parfois, une seule phrase ou quelques mots tombent sur la page suivante --- dans de tels cas, vous pouvez l'utiliser pour améliorer l'effet d'affichage.



05-01-2025    Nuageux       --- Travail à domicile
==================================================

Vous pouvez également organiser votre code comme ceci.


==========
06-01-2025    Ensoleillé    --- Travail à domicile
==========

Ou bien comme ceci...



07/01/2025  Nuageux

  ...



08/01/2025  Nuageux

  Une semaine plus tard, la couleur revient à la précédente...



09/01/2025  Nuageux

  ...


\end{document}
