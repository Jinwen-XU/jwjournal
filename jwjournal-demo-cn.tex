\documentclass[11pt, paperstyle=light yellow, color entry,
  title in boldface, title in sffamily, use style = classical]{jwjournal}
% paperstyle = ... 用于调节纸张颜色,可选项包括:lightyellow、yellow、parchment、green、lightgray、gray、nord、dark
% color entry 可以为每个条目的标题增添更多色彩
% scroll 可以开启卷轴模式,生成类似于长截图的单页 pdf

\UseLanguage { Chinese }

\begin{document}


2025-01-01 晴 | 地点

  在文中出现的日期和标注不会被识别: 2024-12-25、[标注]、【标注】,放心使用!

  [文本]
    你可以使用*斜体*、**粗体**或者***两者结合***(具体效果取决于字体设置)。
    \\
    数学模式下的星号不会被识别:$a^* b^*$。

  【说明】 % 【...】与 [...] 效果相同
    在一些特殊情况需要把星号用 \texttt{\{\}} 隔开:**像**{}***这***{}*样*。

  >>> 像这样可以把文本放在色彩框内。

  图片可以这样引入:

  || <.4> {example-image} % 或者 || {example-image} <.4>
  >> (图片的注释)
  >> (更多注释)

  也可以靠左或者靠右显示:
  (( <.4> {example-image-a} % 或者 (( {example-image-a} <.4>
  -> <.3> % 加入一点竖直间距,例如这里是 0.3\baselineskip
  )) <.4> {example-image-b} % 或者 )) {example-image-b} <.4>


## {新篇章}

2025-01-02
不明(天气也可以写在第二行,看你的喜好)

  标签的颜色会根据日期改变,并且很长的文本会自动缩进:

  [学习] 今天没有学习,但我写了很多很多很多很多很多很多很多很多很多很多很多很多很多很多很多很多代码: `inline code`.

  [注意] 行内代码中的特殊字符需要转义。例如,`\textbackslash` 应写为 `\textbackslash textbackslash`, `\{` 应写为 `\textbackslash\{`, `\%` 应写为 `\textbackslash\%`,等等。

  -----

  三个或更多的\textquote{\texttt{-}}符号被放置在单独的段落中时,就会生成一条分割线 ---

  ---

  --- 不过当前这一行不会被转换,因为在 \texttt{-}\texttt{-}\texttt{-} 之后还有其他文本。

  -----

  如果标注后的文本有多段,可以合理使用 \texttt{\textbackslash\textbackslash} 与 \texttt{\slash\slash} 将它们连接起来,这样所有文本都会被正确缩进:

  [标签] 一些文本。
    //
    更多文本(使用了 \texttt{\slash\slash},因此与上面的文本有一定距离)。
    //
    更多文本(同样与上面的文本有一定距离)。
    //
    (一些注释)
    \\
    (其他注释,由于使用了 \texttt{\textbackslash\textbackslash},和上面是紧贴着的)
    \\
    (更多注释,同样和上面是紧贴着的)



2025-01-03

  [注] 如果只写了日期,那么后面就不会有第二个分隔符。


+++
2025-01-04  多云

  三个加号 \texttt{+++} 可以将当前页面显示的文字多出一行左右,有时候单独的一句话或者几个字落到了下一页,这时可以用它来改善一下显示的效果。



2025-01-05    多云        —— 居家办公
=======================================

也可以把代码排列成这样。


==========
2025-01-06    晴朗        —— 居家办公
==========

或者这样……



2025-01-07  多云

  ...



2025-01-08  多云

  一周之后,又会回到之前的颜色



2025-01-09  多云

  ...


\end{document}
