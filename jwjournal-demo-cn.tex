\documentclass[11pt, paperstyle=light yellow, color entry]{jwjournal}
% paperstyle = ... 用于调节纸张颜色,可选项包括:lightyellow、yellow、parchment、green、lightgray、gray、nord、dark
% color entry 可以为每个条目的标题增添更多色彩
% scroll 可以开启卷轴模式,生成类似于长截图的单页 pdf

\UseLanguage { Chinese }

\begin{document}


2023-01-01 晴 —— 公寓

  在文中出现的日期和标注不会被识别: 2022-02-02、[标注],放心使用

  [体育] 一些关于体育赛事的事情

  [学习] 一些关于学习的事情



2023-01-02
不明(天气也可以写在第二行,看你的喜好)

  标签的颜色会根据日期改变:

  [学习] 今天没有学习,但我写了很多很多很多很多很多很多很多很多很多很多很多很多很多很多很多很多代码。



2023-01-03

  [注] 如果只写了日期,那么后面就不会有第二个分隔符。


+++
2023-01-04  多云

  三个加号 \texttt{+++} 可以将当前页面显示的文字多出一行左右,有时候单独的一句话或者几个字落到了下一页,这时可以用它来改善一下显示的效果。



2023-01-05  多云

  ...



2023-01-06  多云

  ...



2023-01-07  多云

  ...



2023-01-08  多云

  一周之后,又会回到之前的颜色



2023-01-09  多云

  ...


\end{document}
