\documentclass[11pt, paperstyle=light yellow, color entry]{jwjournal}
% paperstyle = ... 用于调节纸张颜色,可选项包括:lightyellow、yellow、parchment、green、lightgray、gray、nord、dark
% color entry 可以为每个条目的标题增添更多色彩
% scroll 可以开启卷轴模式,生成类似于长截图的单页 pdf

\UseLanguage { Chinese }

\begin{document}


2023-01-01 晴 | 公寓

  在文中出现的日期和标注不会被识别: 2022-12-25、[标注]、【标注】,放心使用!

  [体育] 一些关于体育赛事的事情

  【学习】 % 【...】与 [...] 效果相同
  一些关于学习的事情

  图片这可以这样引入:

  || <.4> {example-image-a} % 或者 || {example-image-a} <.4>
  >> (图片的注释)
  >> (更多注释)

  也可以靠左或者靠右显示:
  (( <.4> {example-image-a} % 或者 (( {example-image-a} <.4>
  -> <.3> % 加入一点竖直间距,例如这里是 0.3\baselineskip
  )) <.4> {example-image-a} % 或者 )) {example-image-a} <.4>



2023-01-02
不明(天气也可以写在第二行,看你的喜好)

  标签的颜色会根据日期改变,并且很长的文本会自动缩进:

  [学习] 今天没有学习,但我写了很多很多很多很多很多很多很多很多很多很多很多很多很多很多很多很多代码。

  如果标注后的文本有多段,可以合理使用 \texttt{\textbackslash\textbackslash} 与 \texttt{\slash\slash} 将它们连接起来,这样所有文本都会被正确缩进:

  [标签] 一些文本。
    //
    更多文本(使用了 \texttt{\slash\slash},因此与上面的文本有一定距离)。
    //
    更多文本(同样与上面的文本有一定距离)。
    //
    (一些注释)
    \\
    (其他注释,由于使用了 \texttt{\textbackslash\textbackslash},和上面是紧贴着的)
    \\
    (更多注释,同样和上面是紧贴着的)



2023-01-03

  [注] 如果只写了日期,那么后面就不会有第二个分隔符。


+++
2023-01-04  多云

  三个加号 \texttt{+++} 可以将当前页面显示的文字多出一行左右,有时候单独的一句话或者几个字落到了下一页,这时可以用它来改善一下显示的效果。



2023-01-05    多云        —— 居家办公
=======================================

也可以把代码排列成这样。


==========
2023-01-06    晴朗        —— 居家办公
==========

或者这样……



2023-01-07  多云

  ...



2023-01-08  多云

  一周之后,又会回到之前的颜色



2023-01-09  多云

  ...


\end{document}
