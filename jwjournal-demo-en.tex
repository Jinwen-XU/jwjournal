\documentclass[11pt, paperstyle=light yellow, color entry]{jwjournal}
% "paperstyle = ..." adjusts the paper color, options include: lightyellow、yellow、parchment、green、lightgray、gray、nord、dark
% "color entry" adds more color to the title of each entry
% "scroll" turns on the scroll mode and can generate a single-page pdf similar to a long screenshot

\begin{document}


2023-01-01 Sunny --- Apartment

  Dates and annotations appearing in the text will not be recognized: 2022-02-02, [Annotation], so feel free to write them

  [Sports] something about sporting events

  [Learning] something about learning



2023-01-02
Unknown (the weather can also be written on the second line, depending on your preference)

  The color of the labels will change according to the date.

  [Note] I didn't study today, but I wrote many many many many many many many many many many many many code.



2023-01-03

  [Note] If only the date is written, then there would be no second separator after it.


+++
2023-01-04  Cloudy

  Three plus signs \texttt{+++} can enlarge the current page by about one line. Sometimes a single sentence or a few words fall to the next page --- in such cases, you may use it to improve the display effect.



2023-01-05  Cloudy

  ...



2023-01-06  Cloudy

  ...



2023-01-07  Cloudy

  ...



2023-01-08  Cloudy

  A week later, the color goes back to the earlier one.



2023-01-09  Cloudy

  ...


\end{document}
