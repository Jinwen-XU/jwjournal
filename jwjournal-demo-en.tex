\documentclass[11pt, paperstyle=light yellow, color entry, month-day-year,
  title in boldface, title in sffamily, use style = classical]{jwjournal}
% "paperstyle = ..." adjusts the paper color, options include: lightyellow、yellow、parchment、green、lightgray、gray、nord、dark
% "color entry" adds more color to the title of each entry
% "scroll" turns on the scroll mode and can generate a single-page pdf similar to a long screenshot

\begin{document}


01/01/2024 Sunny | Location

  Dates and annotations appearing in the text will not be recognized: 12/25/2023, [Note], so feel free to write these.

  [Text]
    You can write text in *italics*, **bold** or ***both***.
    \\
    Asterisks in math mode will not be recognized: $a^* b^*$.

  [Remark] In some special cases, it would be necessary to separate the asterisks with~\texttt{\{\}}: **like**{}***this***{}*one*.

  >>> You can put text inside a colored block like this.

  Images can be included via:

  || <.4> {example-image} % or || {example-image} <.4>
  >> (Remark on the image)
  >> (More remark...)

  +++
  You may also show the image on the left or on the right:
  (( <.4> {example-image-a} % or (( {example-image-a} <.4>
  -> <.3> % add some vertical spacing, here is 0.3\baselineskip
  )) <.4> {example-image-b} % or )) {example-image-b} <.4>


## {\textsc{A New Section}}

01/02/2024
Unknown (the weather can also be written on the second line, depending on your preference)

  The color of the labels shall change according to the date, and long text shall be automatically wrapped and indented:

  [Note] I didn't study today, but I wrote many many many many many many many many many many many many code: `inline code`.

  [Attention] Special characters in inline code need to be escaped. For example, `\textbackslash` should be written as `\textbackslash textbackslash`, `\{` should be written as `\textbackslash\{`, `\%` should be written as `\textbackslash\%`, etc.

  -----

  Three or more \textquote{\texttt{-}} in a separate paragraph give you a separation line ---

  ---

  --- but the current line won't be matched since there is extra text behind \texttt{-}\texttt{-}\texttt{-}.

  -----

  If there are multiple paragraphs following the label, you may use \texttt{\textbackslash\textbackslash} and \texttt{\slash\slash} wisely to concatenate them, so that all texts would be indented properly:

  [Label] Some text.
    //
    More text (with \texttt{\slash\slash}, thus having some vertical spacing from the above text).
    //
    More text (same, with some vertical spacing from the above text).
    //
    (Some remark)
    \\
    (Another remark, due to the use of \texttt{\textbackslash\textbackslash}, is close to the above remark)
    \\
    (More remark --- similarly, is close to the above remark)



01/03/2024

  [Note] If only the date is written, then there would be no second separator after it.


+++
01/04/2024  Cloudy

  Three plus signs \texttt{+++} can enlarge the current page by about one line. Sometimes a single sentence or a few words fall to the next page --- in such cases, you may use it to improve the display effect.



01-05-2024    Cloudy        --- Work at home
============================================

You may also organize your code like this.


==========
01-06-2024    Sunny         --- Work at home
==========

Or like this...



01/07/2024  Cloudy

  ...



01/08/2024  Cloudy

  A week later, the color goes back to the earlier one.



01/09/2024  Cloudy

  ...


\end{document}
