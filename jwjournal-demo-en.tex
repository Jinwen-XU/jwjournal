\documentclass[11pt, paperstyle=light yellow, color entry, month-day-year]{jwjournal}
% "paperstyle = ..." adjusts the paper color, options include: lightyellow、yellow、parchment、green、lightgray、gray、nord、dark
% "color entry" adds more color to the title of each entry
% "scroll" turns on the scroll mode and can generate a single-page pdf similar to a long screenshot

\begin{document}


01/01/2023 Sunny | Apartment

  Dates and annotations appearing in the text will not be recognized: 12/25/2022, [Note], so feel free to write these.

  [Sports] something about sporting events.

  [Learning] something about learning.



01/02/2023
Unknown (the weather can also be written on the second line, depending on your preference)

  The color of the labels will change according to the date.

  [Note] I didn't study today, but I wrote many many many many many many many many many many many many code.



01/03/2023

  [Note] If only the date is written, then there would be no second separator after it.


+++
01/04/2023  Cloudy

  Three plus signs \texttt{+++} can enlarge the current page by about one line. Sometimes a single sentence or a few words fall to the next page --- in such cases, you may use it to improve the display effect.



==========
01-05-2023    Cloudy        --- Work at home
==========

You may also organize your code like this.


==========
01-06-2023    Sunny         --- Work at home
==========

...



01/07/2023  Cloudy

  ...



01/08/2023  Cloudy

  A week later, the color goes back to the earlier one.



01/09/2023  Cloudy

  ...


\end{document}
