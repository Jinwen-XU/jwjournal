\documentclass[11pt, paperstyle=light yellow, color entry, day-month-year,
  title in boldface, title in sffamily, use style = classical]{jwjournal}
% "paperstyle = ..." passt die Papierfarbe an, Optionen sind: lightyellow、yellow、parchment、green、lightgray、gray、nord、dark
% "color entry" fügt dem Titel jedes Eintrags mehr Farbe hinzu
% "scroll" schaltet den Scroll-Modus ein und kann ein einseitiges PDF ähnlich einem langen Screenshot generieren

\UseLanguage { German }

\begin{document}


01.01.2024 Sonnig | Ort

  Im Text erscheinende Daten und Anmerkungen werden nicht erkannt: 25.12.2023, [Anmerkung], können Sie diese ruhig schreiben.

  [Text]
    Sie können Text in *kursiv*, **fett** oder ***beides*** schreiben.
    \\
    Sternchen im Mathematikmodus werden nicht erkannt: $a^* b^*$.

  [Anmerkung] In einigen Sonderfällen kann es notwendig sein, die Sternchen mit~\texttt{\{\}} zu trennen: **genau**{}***wie***{}*dieses*.

  >>> Sie können Text in einen farbigen Block wie diesen einfügen.

  Bilder können eingefügt über:

  || <.4> {example-image} % oder || {example-image} <.4>
  >> (Anmerkung zum Bild)
  >> (Weitere Bemerkung...)

  +++
  Sie können das Bild auch links oder rechts anzeigen:
  (( <.4> {example-image-a} % oder (( {example-image-a} <.4>
  -> <.3> % fügen Sie etwas vertikalen Abstand hinzu, hier ist „0.3\baselineskip“
  )) <.4> {example-image-b} % oder )) {example-image-b} <.4>


## {\textsc{Ein neuer Abschnitt}}

02.01.2024
Unbekannt (das Wetter kann je nach Wunsch auch in die zweite Zeile geschrieben werden)

  Die Farbe der Beschriftungen ändert sich je nach Datum und langer Text wird automatisch umbrochen und eingerückt:

  [Hinweis] Ich habe heute nicht gelernt, aber ich habe sehr, sehr, sehr, sehr, sehr, sehr, sehr, sehr, sehr, sehr, sehr viel Code geschrieben: `Inline-Code`.

  [Achtung] Sonderzeichen im Inline-Code müssen maskiert werden. Beispielsweise sollte `\textbackslash` als `\textbackslash textbackslash` geschrieben werden, `\{` sollte als `\textbackslash\{` geschrieben werden, `\%` sollte als `\textbackslash\%` geschrieben werden usw.

  -----

  Drei oder mehr \textquote{\texttt{-}} in einem separaten Absatz ergeben eine Trennlinie ---

  ---

  --- aber die aktuelle Zeile wird nicht abgeglichen, da sich hinter \texttt{-}\texttt{-}\texttt{-} zusätzlicher Text befindet.

  -----

  Wenn auf die Beschriftung mehrere Absätze folgen, können Sie \texttt{\textbackslash\textbackslash} und \texttt{\slash\slash} verwenden, um sie zu verketten, sodass alle Texte richtig eingerückt werden:

  [Beschriftung] Etwas Text.
    //
    Mehr Text (mit \texttt{\slash\slash}, also mit etwas vertikalem Abstand zum obigen Text).
    //
    Mehr Text (dasselbe, mit etwas vertikalem Abstand zum obigen Text).
    //
    (Einige Anmerkungen)
    \\
    (Eine weitere Anmerkung, die aufgrund der Verwendung von \texttt{\textbackslash\textbackslash} direkt neben der obigen Anmerkung)
    \\
    (Weitere Anmerkungen --- ähnlich, direkt neben der obigen Anmerkung)



03.01.2024

  [Anmerkung] Wenn nur das Datum geschrieben wird, dann gäbe es kein zweites Trennzeichen danach.


+++
04.01.2024  Bewölkt

  Drei Pluszeichen \texttt{+++} können die aktuelle Seite um etwa eine Zeile vergrößern. Manchmal fallen ein einzelner Satz oder ein paar Wörter auf die nächste Seite --- in solchen Fällen können Sie es verwenden, um den Anzeigeeffekt zu verbessern.



05.01.2024    Bewölkt       --- Arbeit zu Hause
===============================================

Sie können Ihren Code auch so organisieren.


==========
06.01.2024    Sonnig        --- Arbeit zu Hause
==========

Oder so...



07.01.2024  Bewölkt

  ...



08.01.2024  Bewölkt

  Eine Woche später wird die Farbe wieder auf die vorherige zurückgesetzt.



09.01.2024  Bewölkt

  ...


\end{document}
